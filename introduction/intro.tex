As our society moves toward digitalization of the world around us and increasing reliance on software application, computerized database systems have became prevalent and important. As the number of software applications and services (such as Facebook and Gmail) grew, so are the amount of data we generate. Databases management systems (DBMS) are the core of most software applications; therefore, the efficiency of the DBMS are crucial to the applications they support. Here we distinguish software applications and DBMS as follows: applications are the software services that consists of the logic of the software application and how the data are consumed; on the other hand, a database or DBMS has very little logic. The purpose of DBMS is to store and efficiently retrieve data; and that is it. A database may be able to support multiple applications.

The earliest form of databases are physical written records on paper or other forms. Computerized databases are also not a new concept. Since the earliest computer systems, software applications have always had the need to save and store data; therefore, a computer file is essentially a database if the file can be consistently read and stored. The earliest DBMS utilized the hierarchical data model\cite{hdbms} that was first described in the 1960s and implemented by the IBM Information Management System (IMS)\cite{ims, intro_ims}. The relational model\cite{DBLP:journals/cacm/Codd70} was described by Edgar Codd in 1970, and remained one of the most prevalent database model in the software industry. The relational model is the basis for popular database management systems such as MySQL, Microsoft SQL Server, Oracle Database, and SQLite.

Despite the fact that database management systems have been common place in software projects, there are rooms for improvement in specific application domains. The growth of volume and diversity of data on the Web has also ushered in challenges that motivated new flexible data models, such as the Resource Description Framework (RDF) for semantic web data. This dissertation addresses three application domains and operations in which through data indexing techniques, the query efficiency can be improved. The addressed application domains are skyline computation, hierarchical data in relational databases, and pointed-based spatial data (POIs) in RDF triple stores.

This dissertation is organized as follows. The rest of this chapter provides an introduction to the chapters covered in this dissertation. Starting from chapter two, each application domain is discussed in detail, where each chapter is structured as a standalone discussion with introduction, related works, and discussion of the technique. Chapter~2 covers Skyline computation in broadcast environments; chapter~3 discusses an index technique for hierarchical data in relational databases; chapter~4 describes Geo-Store: a spatially-aware RDF triple store and SPARQL query engine; chapter~5 concludes this dissertation and provides future work; and Appendix~A provides a case study of computerized control systems for scientific and industrial scale experiments and describe the current state of the control system of the MDPX experiment. This appendix also shows that many of the topics covered in this dissertation may be used in practice.

\section{Skyline Evaluation in Wireless Data Broadcast Environments}

Given a data set $T$, the skyline is an operation that find a subset $S$ of $T$ that are the best or the most efficient in $T$, where `the most efficient' is defined by the user of the query. The proper definition for the skyline operator is that for a record, $s$, to be in $S$, the record $s$ must not be dominated by other records in the original data set $T$\cite{shooting_stars, progressive_skyline}. For example, someone who is planning for an ocean-view vacation would be interested in finding a list of hotels that are close to the ocean and at the same time not too expensive. In this case, the user preference for our query is to minimize both the distance to the beach and the cost of the hotel. If a hotel $a$ is the same distance to the beach as $b$, but has higher cost, then $a$ is dominated by $b$; therefore, $a$ cannot be in $S$.

Skyline computer in broadcast environment is described in chapter~2, which is divided into two parts. The first part describes the constraints in data broadcast environments and proposes tree-based distributed index (TDI) an data serialization technique to address these constraints. The second part of the chapter describes our skyline computation technique, index-based pruning skyline (IPS). In IPS, we construct an data index of the data set $T$ using R-Tree spatial index. After that, IPS computes skylines by aggressively prune the search space as the index is traversed by the search algorithm. Because IPS prune the search earlier than record-based pruning techniques, our algorithm performs better.

\section{Hierarchical Data in Relational Database Management Systems}

Relational database are one of the most prevalent category of database management systems used in the software industry due to its performance, and standardized query language SQL. However, the relational model is not optimized for storing data that models hierarchical relationship. Microsoft has recently announced the support of hierarchical data in Transact-SQL starting in SQL Server 2008\cite{expert_sql_server}; however, there are limited to no support in other systems.

The simplest and most intuitive way to represent hierarchy in relational databases is using the adjacency list model. In this model, every record in the hierarchical relation (table) has a pointer to the parent record in the same relation. One drawback of adjacency list model is that some hierarchical queries and operations require recursion or iteration. Having to perform iterations means that our application has to submit multiple queries to the database server to accomplish the operation.

In chapter~3, we describe another model to store hierarchy in relational databases based on nested sets structure and is an extension to the adjacency list model. In our approach, we add two additional fields, left and right, are augmented to each record to denote the hierarchical relationship. The two fields is then indexed in B-Tree to allow querying relationship using only index scans. In the chapter we first describe the hierarchical operations we use in the chapter; then, we describe the storage and index technique. Our expectation of our design can remove recursions in many of the operations and therefore improve query turnaround time.

\section{Geo-Store: A Spatially-Aware SPARQL Evaluation Engine}

As the data of an application becomes more diverse or originates from multiple sources, a more schema-relaxed data model is desired to simplify the process of data integration. The Resource Description Framework (RDF) is a schema-relaxed data storage model that is used to store the diverse data on the Web. RDF stores data as a set of statements consists of three parts: subject, predicate, and object. The subject is the entity being described; predicate is the property of the subject; and object is the value of the property. For this reason, RDF databases are often called triple stores. Using this format, an RDF database can have loose schema or completely schema free.

In chapter~4 of this dissertation, we specifically focus on point-based spatial data (POIs) in RDF triple stores. Our contribution in this chapter are three-fold. The first is to define range and k-nearest-neighbor query filter in SPARQL. The second is a unique spatial data index technique utilizing the Hilbert Space Filling Curve (HSFC). The third is an implementation of our query filters and index techniques our Geo-Store triple store. 