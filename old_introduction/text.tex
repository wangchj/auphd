\chapter{Preliminaries}

\section{Introduction}
Since its introduction by Edgar Codd in 1970 \cite{DBLP:journals/cacm/Codd70}, the relational data model has become the dominant model for computerized database management systems (DBMS). Due to its flexibility in modeling data relationships, relational model has large replaced older data storage models, such as the hierarchy model and the network model, in which relationships are rigidly defined. Due to the amount of data we generate, relational database management system (RDBMS) has become an indispensable part of most new software projects and systems \footnote{http://www.gartner.com/it/page.jsp?id=507466}. Well-known relational database management systems include MySQL, Microsoft SQL Server, PostgreSQL, SQLite just to name a few.

In relational model, data is stored in tables (also known as relations) within a database. Each table is composed of a set of attributes (or fields) and stores a set of data with these common attributes. For example, an RDBMS for a human resource system may contains a table named Employees. This table would contain the attributes name, date of birth, department, and position and store the employees within a company and each entry (or tuple) of the table would represent the information for an unique employee in the system. The database may contain additional table, such as Departments and Pay Brackets, that completes the database.

One of the benefits of the relational model over older database storage models is that the relationships of the records stored in the database are not rigidly defined. The entries (or records) in a table are not implicitly link to any other records, unlike older data storage models \footnote{http://en.wikipedia.org/wiki/Hierarchical\_database\_model}. Relationships between records are also stored in tables, as data, and can be queried. This flexibility allows the database to be free from a single application, and to serve multiple purposes. The tables in a database can be joined at runtime to satisfy particular query.

Despite its wide adoption, the relational data model (and therefore relational systems) is not without shortcomings. One of the most obvious shortcoming is the rigid table structure in such a way that every entry of a table has to fit into the fields of the table. For example, the Employees table has 4 attributes. But what if an employee does not have all the attributes or an employee requires a much larger number of fields to be properly recorded? We could defined a table with a long list of attributes to satisfy every entry but the table may have many missing (null) values and contribute to space and performance overhead\cite{DBLP:conf/icde/BeckmannHKN06}. A more encouraged way is to normalize \cite{DBLP:journals/cacm/Codd70} (or divide) the long list of attributes into separate tables and each table only stores a number of cohesive attributes that often reflect reusable (and perhaps real-world) entities. Normalization mitigates sparse table, but creates fragmentation of information resulting in complicated data schema\cite{DBLP:conf/sigmod/ChuBN07}.

Data change is inevitable in scientific database due to initial incomplete knowledge of the domain of study and the evolution of study. In scientific database, data is constantly change not not only in terms of increasing data records, but also in the form of evolution of schema (the kind of data to be recorded) over time. In a relational database, schema changes leads to sparse, and very wide tables. If we try to reorganize the database through normalization, we may end up with a database with a significantly different table names and structure, which leads to significant rewrite of the application that uses the database. [TODO: example and illustration here?]

To avoid complex database schemas that commonly associate with relational databases, many scientific databases use their own data format.

In addition to scientific database, we are witnessing a tremendous growth of information on the Web\footnote{http://news.netcraft.com/archives/2013/01/07/january-2013-web-server-survey-2.html}. Like scientific databases, the Web is an ever expanding, dynamic (always changing) source of information, with the expansion rate that is faster than any localized database. In addition to be rapidly expanding, the Web is also an enormous source of semi-structured to unstructured data. Lets take an online retailer, like Amazon, as an example. For its vast amount of merchandizes, there are broad categories (such as books, clothing, shoes, electronics), but each item in each category is different. A shirt has a set of attributes, a different shirt may have another set of attribute, and there is no uniform set of attributes that capture variation of all shirts. The information on the Internet is so vast, a database of rigid table structure is insufficient.

In recent years there is a movement of designing an information storage model that is flexible enough for the dynamic and unstructured data. Several proposed models are discussed in section [A]. The most significant model (to this writing) is the movement of the semantic web. The purpose of the semantic web is to index and add meaning to the information on the web. This is an ambitious endeavour indeed, and there are several proposed standards (protocols) at play. One of the standards is a storage model called Resource Description Framework (RDF), which allows very flexible data storage. More in depth discussion of the Semantic Web and RDF will be discussed in section [B].

The goal of work composes of two parts: the first is to study the semantic web in the setting of scientific experimental database, and the second is to study optimizations of spatial data management and application in semantic web data storage model. For the first goal, we will propose the benefit of using semantic web technology over traditional relational database for complex scientific database and validate our proposal with a case study. For our second goal, we will study current spatial data management techniques and apply them to semantic web data stores.

[Need an outline here]

\section{Related Works}

\begin{enumerate}
  \item Current work in comparing and converting RDF to relational database
  \item Other NoSQL database designs used to solved the problem of indexing the web. For example: Cassandra, BigTable
  \item Related work for spatial database?
\end{enumerate}