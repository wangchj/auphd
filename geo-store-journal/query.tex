
The SPARQL query language is standardized by the World Wide Web
Consortium (W3C) for querying RDF data. Evaluation of SPARQL
queries is based on pattern matching on the target RDF graph. A
pattern may contain variables that are bound to a URI or a
literal. In addition, the SPARQL query language provides a number
of built-in FILTER functions, which can be easily extended to
support spatial operations (or constraints). Because range and $k$
nearest neighbor queries are the building blocks of location-based
services, we elaborate on how to evaluate the two important query
types by utilizing our system in this section.


\subsection{Range Queries}


Listing~\ref{list5} shows a sample range query which is launched
to retrieve the latitude and longitude values of all the qualified
gas stations with the two selection conditions: (1) the location
is less than 10 miles away from the query point (the mobile user's
current location), and (2) the gas price is cheaper than \$3.00
per gallon.

%\newpage

\lstset{showstringspaces=false} \lstset{basicstyle=\ttfamily\small,escapeinside={`}{`}}

\begin{lstlisting}[caption={The sample range query in Geo-Store (use case 1).},
                   label={list5},frame=single]
SELECT ?coordinates
WHERE { ?point  a           "Gas Station" .
        ?point  gml:pos     ?coordinates .
        FILTER  (gstore:range(?point, CURRENT_LOCATION, 10, "mile"))
        ?point  gstore:gaspriceUSD       ?price .
        FILTER  (?price < 3.00)
      }
\end{lstlisting}

\begin{figure*}[!h]
\begin{center}
 \begin{tabular}{cc}
 \psfig{figure=geo-store-journal/image/hilbert.eps,height=2.0in}  &
 \psfig{figure=geo-store-journal/image/range_query.eps,height=2.0in}  \\
 \parbox{2.2in}{\centering (a) Hilbert curve transformation.} &
 \parbox{2.2in}{\centering (b) Range query $Q_R$.} \\
 \end{tabular}
 \caption{Hilbert curve transformation and range query.}
 \label{fig-hilbert}
\end{center}
\end{figure*}

In Geo-Store, each spatial object is annotated with its Hilbert
value information based on the GeoHilbert representation.
Figure~\ref{fig-hilbert}(a) illustrates an example of mapping
spatial objects in a two dimensional space into their Hilbert
values. In Figure~\ref{fig-hilbert}(a), the entire space is
divided by the Hilbert curve into 64 grids with their unique
Hilbert values, and we can acquire the Hilbert values of the
spatial objects \emph{A}, \emph{B}, and \emph{C}, as 29, 32, and
7, respectively. Depending on the desired resolution, more
fine-grained curves can be recursively generated based on the
Hilbert curve generation algorithm. Figure~\ref{fig-hilbert}(b)
demonstrates how a range query can be processed in our system by
taking advantage of the Hilbert values of spatial objects. As
depicted in Figure~\ref{fig-hilbert}(b), the query point is
\emph{q} and the query window of the range query $Q_R$,
highlighted in red, covers the three Hilbert curve segments
[10-12], [17-18], and [28-31]. After obtaining the above three
curve segments, our system retrieves all the spatial objects whose
Hilbert values are embraced by the three curve sections and treats
the retrieved spatial object set $\mathbb{R}^{\prime}$ as the
inclusive query result. Subsequently, our system examines all the
spatial objects in the grids that \emph{partially} overlap with
the query window (i.e., grids [10-12], [28] and [31]) to check if
their exact locations (i.e., latitude and longitude values) are
within the query window by dictionary lookup. Finally, the exact
query result $\mathbb{R}$ is returned to the user after filtering
out those objects whose locations are outside the query window in
the partially overlapping grids.


\subsection{$k$ Nearest Neighbor Queries}

In this subsection, we extend our range query solution to evaluate
$k$ nearest neighbor queries efficiently in Geo-Store.
Listing~\ref{list6} demonstrates a sample $k$ nearest neighbor
query which is issued to retrieve the latitude and longitude
values of the gas stations that are among the $k$ closest gas
stations to the query point with a listed gas price cheaper than
\$3.00 per gallon.


\lstset{showstringspaces=false} \lstset{basicstyle=\ttfamily\small,escapeinside={`}{`}}

\begin{lstlisting}[caption={The sample $k$ nearest neighbor query in Geo-Store (use case 2).},
                   label={list6},
				   frame=single]
SELECT ?coordinates
WHERE { ?point  a           "Gas Station" .
        ?point  gml:pos     ?coordinates .
        FILTER  (gstore:NN(?point, CURRENT_LOCATION, k))
        ?point  gstore:gaspriceUSD       ?price .
        FILTER  (?price < 3.00)
      }
\end{lstlisting}

Given a query point $q$, Geo-Store searches the spatial objects in
both the ascending and descending directions of Hilbert values
until $k$ spatial objects are found, and then records the result
set as $\mathbb{S}$. Supposing the object $o$ has the longest
distance to $q$ in $\mathbb{S}$, $Distance(q, o)$ (the distance
between $q$ and $o$) is set as the \emph{search upper bound} for
the subsequent range query. Afterwards, Geo-Store launches a range
query $Q_{R}$ with $Distance(q, o)$ to decide the query window
size and then acquires the query result $\mathbb{R}^{\prime}$.
Next, Geo-Store identifies the top $k$ objects in
$\mathbb{R}^{\prime}$ based on their respective distances to $q$
in order to derive the final query result $\mathbb{R}$.

Figure~\ref{fig-NN} demonstrates the evaluation of a $k$ nearest
neighbor query with Geo-Store. As shown in Figure~\ref{fig-NN}(a),
based on the query point $q$, Geo-Store first searches for spatial
objects with Hilbert values $\geq 30$ and $< 30$ in parallel until
$k$ objects are discovered. Next, assume that the spatial object
that has the longest distance to $q$ among the above $k$ objects
is object $B$. Then, a range query $Q_R$ with the distance between
$q$ and $B$ as the search upper bound is issued; $Q_R$ returns the
result set $\mathbb{R}^{\prime}$, as depicted in
Figure~\ref{fig-NN}(b). In this example, set $\mathbb{R}^{\prime}$
encompasses objects which fall on the five Hilbert curve segments,
[8-20], [23-24], [27-32], [35-36], and [53-54]. Finally, Geo-Store
computes the top $k$ objects in $\mathbb{R}^{\prime}$ with the
shortest distance to $q$ as the exact query result.

\begin{figure*}[!h]
\begin{center}
 \begin{tabular}{cc}
 \psfig{figure=geo-store-journal/image/NN_search.eps,height=2.0in}  &
 \psfig{figure=geo-store-journal/image/NN.eps,height=2.0in}  \\
 \parbox{2.3in}{\centering (a) Finding the search upper bound.} &
 \parbox{2.3in}{\centering (b) The derived range query $Q_{R}$.}
 \end{tabular}
 \caption{$k$ nearest neighbor query example.}
 \label{fig-NN}
\end{center}
\end{figure*}
