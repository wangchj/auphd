
With the popularity of wireless networks and mobile devices,
Location-Based Services (LBS) have become indispensable
applications to mobile users. The global GPS navigation and LBS
market size has been predicted to grow significantly from \$1.6
billion in 2009 to \$13.4 billion in 2014 according to IEMR's
report ``Global GPS Navigation and Location Based Services
Forecast". In addition, the Web consists of a huge volume of data
which requires the use of human intelligence to process and analyze. The main goal of the
Semantic Web is to augment the Web with information so that
computers can understand and exchange Web data. Consequently, it
will be a trend for LBS providers to employ Semantic Web data
sources to develop semantics-enabled location-based services.

The Resource Description Framework (RDF) data model is designed
for interchanging schema-relaxable (or schema-less) data on the
Semantic Web~\cite{RDF}. RDF models the linking structure of the
Web as triples of the form $\langle s, p, o \rangle$, where $s$ is
a subject, $p$ is a predicate, and $o$ is an object. Each triple
represents the relationship between the subject and the object. A
collection of triples forms a directed graph, where the edges
represent predicates between subjects and objects, which are
represented by the graph nodes. With the increasing amount of RDF
data on the Web, researchers developed specialized architectures
for RDF data management named \emph{triple
stores}~\cite{conf/www/CarrollDDRSW04,conf/vldb/ChongDES05,conf/vldb/AbadiMMH07,
journals/pvldb/WeissKB08,journals/vldb/NeumannW10}. Generally,
these solutions employ various indexing, compression, and query
optimization techniques for scalable and efficient management of
RDF data.

The Semantic Web is an ideal data source for supporting the
state-of-the-art location-based services that employ dynamic or
near real-time information. For example, a mobile user may utilize
LBS to search for nearby restaurants based on recent reviews on
the Web. However, to the best of our knowledge, there are limited
solutions to process spatial queries, the building blocks of most
LBSs, on triple stores for supporting advanced location-based
services. Therefore, the goal of the Geo-Store project is to
develop novel spatial query techniques that are able to
efficiently evaluate spatial queries on RDF triple stores for
providing semantics-enabled location-based services. The main
features of our Geo-Store system are as follows.

\begin{itemize}
    \item The Geo-Store system employs a novel representation to
    model spatial features and utilizes a spatial mapping
    mechanism to preserve spatial locality.\vspace*{-3pt}

    \item The Geo-Store system is able to effectively process both range and $k$ Nearest Neighbor ($k$NN)
    queries, the building blocks of many LBS applications, on RDF triple stores.\vspace*{-3pt}

    \item Users are able to integrate and operate the Geo-Store system on
    existing triple stores (e.g., RDF-3X) with limited changes.\vspace*{-3pt}
\end{itemize}
