Location-based services are any service that takes into account
the geographic location of an entity and are accessible with
mobile devices through wireless
networks~\cite{journals/cacm/JunglasW08}. With the prevalence of
GPS-enabled mobile devices and the introduction of 4G mobile
telecommunications services, various commercial LBSs, such as
location-based dating, location-targeted advertisement, and child
safety services, appear in our
lives~\cite{journals/pervasive/BellavistaKH08}. In addition, novel
LBS applications are able to exploit online Semantic Web sources
(e.g., LinkedGeoData\footnote{http://linkedgeodata.org}) about
nearby physical entities of a user to provide personalized
services~\cite{journals/internet/WoenselCPT11}. Today,
location-based services are applied in different fields, such as
emergency response, navigation, product tracking, social networks,
etc.

%The origin of LBS was the Enhanced 911 (E911) mandate which was
%issued by the US Federal Communication Commission (FCC) to improve
%emergency responses to wireless 911 calls by determining a
%caller's location with prescribed accuracy.

%\footnote{http://www.w3.org/standards/semanticweb/}

The Semantic Web is a group of techniques for machines to
understand information and exchange knowledge on the World Wide
Web. The cornerstone of the Semantic Web is a logical data model
named RDF which employs triples to represent the relationships
between subjects and objects. In order to efficiently manage RDF
data, there are numerous systems invented for storing and querying
triple
collections~\cite{conf/www/CarrollDDRSW04,conf/vldb/ChongDES05,conf/vldb/AbadiMMH07,
journals/pvldb/WeissKB08,journals/vldb/NeumannW10}. For improving
performance and scalability, Abadi et
al.~\cite{conf/vldb/AbadiMMH07} introduced a solution by
vertically partitioning the RDF data. Their solution's performance
can be further improved by utilizing a column-oriented DBMS, which
is a database designed specially for the vertically partitioned
case. Weiss et al.~\cite{journals/pvldb/WeissKB08} proposed a
sextuple-indexing scheme, named Hexastore, which allows for quick
and scalable general purpose query evaluation for RDF data
management. Hexastore achieves significant advantages in
performance compared with previous solutions for managing RDF
triples. The RDF-3X engine~\cite{journals/vldb/NeumannW10} is an
implementation of the SPARQL query language~\cite{SPARQL} for RDF
by pursuing a simplified architecture with streamlined indexing
and query processing. The design of RDF-3X completely eliminates
the need for index tuning by exhaustive indexes for all
permutations of subject-predicate-object triples and their binary
and unary projections. However, all the aforementioned triple
stores do not consider the unique features of spatial data when
they encode and store RDF triples.

%In addition, the query processor is able to leverage fast merge
%joins for improving query performance.

Recently, Perry et al.~\cite{Perry11} presented an extension to
SPARQL, named SPARQL-ST, for complex spatiotemporal queries. They
implemented SPARQL-ST by extending a relational database, which is
not an effective way of managing RDF triples. The Strabon system
is an implementation of the data model stRDF and the query
language stSPARQL~\cite{conf/esws/KoubarakisK10}, which are
extensions of RDF and SPARQL for managing spatial and temporal
data. Nevertheless, Strabon is built on top of the RDF store
Sesame\footnote{http://www.openrdf.org/} which is not as efficient
as the aforesaid new generation RDF triple
stores~\cite{KyzirakosKK10}. Brodt et
al.~\cite{conf/gis/BrodtNM10} proposed a solution to integrate
spatial query processing into RDF triple stores. However, their
design cannot efficiently evaluate queries on large-scale spatial
data sets, which are essential for the state-of-the-art LBSs.
Furthermore, Parliament~\cite{conf/SSWS09,conf/semweb/KolasS07} is
a triple store and employs a similar approach
to~\cite{conf/gis/BrodtNM10} for spatial queries and storage of
the data. The GeoSPARQL draft standard~\cite{GeoSPARQL} is
currently being implemented in Parliament.

%Therefore, we need novel techniques to manage spatial RDF data and
%efficiently enable semantics-enhanced location-based services.
