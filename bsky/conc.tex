In this work, we presented a technique for efficient skyline computation in broadcast environment. To make the broadcast program self-descriptive, we used an R-Tree to index broadcast data records. We then introduced Tree-Based Distributed Index (TDI) program allocation, in which the index and data records are serialized and pushed to the broadcast channel.

In addition, we presented two algorithms (RPS and IPS) for evaluating skyline results from TDI broadcast program. Ours is only work we know that is able to evaluate skyline for data record with arbitrary dimensions and arbitrary skyline preferences.

Our simulation results with synthetic data show that our program allocation technique create program with low index overhead. In addition, our skyline algorithms performs well with variety of data-sets.

%In this paper, we presented a broadcast data stream allocation technique (TDI) that utilizes the R-tree and performs a depth-first traversal of the index to create a distributed index. The goals of TDI are to facilitate query processing from broadcast data, to reduce the index overhead (IP) and to improve the initial index probe. TDI is able to achieve all these with reasonable efficiency. TDI is a flexible R-tree based index and supports skyline queries as well as other data query types. The allocation distributes $b^h$ number of index segments among the broadcast program to reduce the initial index, and at the same time keeps the index overhead low. The simulation results for index percentage show that TDI performs very well with two levels of replication and follow the efficiency of one-time index with only a $16\%$ increase in index overhead.

%The simulation results show that the approach also performs well with data of higher dimensions. The index overhead decreases as the number of records remain constant and the number of data dimensions increase. This is due to the fact that the growth of dimensionality does not make the index tree grow ``taller" and does not incur the cost of new nodes when the index grows. The height of an index tree does increase as the number of records increase, but as seen in Figure~\ref{fig:ip_rc}, the growth of the index is not as fast as the growth of the amount of data; therefore, the index overhead decreases as records increase.

%In addition, we introduced point-based and index-based pruning skyline algorithms. The experiments show that both algorithms are capable of evaluating skyline queries of combined $min$ and $max$ attributes with reasonable tuning time and dominance tests. The index-based skyline has always performed better than the point-based skyline, in some cases by several factors. The performance of the algorithms is also affected by the data arrangement and R-tree implementation. From our simulation, we find that R-trees that index records with lower attributes first perform better for $min$ skyline queries, and vice versa.