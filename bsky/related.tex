
\subsection{Skyline Computation}

Algorithms of the computation of skyline records in traditional database systems have been studied in~\cite{conf/icde/BorzsonyiKS01,shooting_stars,progressive_skyline}. The well-known algorithms are Nested-Loop, Block-Nested-Loop, and Divide and Conquer have studied in \cite{conf/icde/BorzsonyiKS01}. The Nested-Loop algorithm is a naive algorithm that it compare every record with every other record. In Block-Nested-Loop, a record is only compared with other records in the same block. In Divide and Conquer (D\&C) utilizes divide and merge strategy to computer skyline.

%\cite{shooting_stars} introduces an online skyline computation algorithm in which the skyline are computed progressively. The first skyline is returned almost immediately and more skyline points are added to the result set.

Branch-and-Bound Nearest Neighbor \cite{progressive_skyline} algorithm utilizes a tree index to efficiently computes skyline. In the design, data records are represented geometrically and a tree index is built on the data. The algorithm then uses branch-and-bound tree traversal to find nearest neighbors as skyline. Our algorithms uses a geometric tree index, but BBS-NN cannot be easily adopted to the broadcast environment due to that the algorithm requires backtracking of the tree.

%Nearest Neighbor (NN) and Branch-and-Bound (BBS) skyline algorithms are two of the best-performing algorithms for progressive skyline computation for traditional database systems presented in~\cite{progressive_skyline}. In the NN skyline algorithm, the records of the data set are presented geometrically in a Euclidean space with the relevant attributes as the coordinate in each dimension. In the example of a hotel close to the ocean previously presented, a record would be placed on a plane with the distance of the hotel to the ocean and its price as the two axes that determine its ranking, and the values in the two attributes as the coordinates of the records on the Euclidean plane. The NN algorithm finds a nearest neighbor from the axes; that is the first skyline point. Then the algorithm marks the region that is dominated by the first skyline point so that anything in that region will not be searched. The search results in two more search regions and the search continues until no more records are left.

%as in Figure~\ref{fig:skyline_nn}

%\begin{figure}[h]
%\begin{center}
%\includegraphics[width=2in]{Figures/skyline_nn.eps}
%\vspace*{-5pt} \caption{NN d and its pruning region.}
%\label{fig:skyline_nn} \vspace*{-10pt}
%\end{center}
%\end{figure}

%BBS is an algorithm that surpasses the computation efficiency of the NN algorithm. BBS utilizes the best first search technique to traverse the index tree to prune the unnecessary branches. Unfortunately, both NN and BBS cannot be easily adopted to the broadcast environment due to the linear nature of the broadcast program. Both NN and BBS require backtracking of the index tree to find the best path to prune (which is impermissible in the broadcast environment) or require waiting for the next broadcast cycle (which incurs a long waiting time). The solutions presented in this chapter will use the pruning region strategy used in NN. Our algorithms systematically build pruning regions, as presented in the NN algorithm, as the client receives and discovers more data from the broadcast channel.

% SQL Operator
Extension SQL syntax for skyline was defined by B{\"o}rzs{\"o}nyi et al. in~\cite{conf/icde/BorzsonyiKS01}. The syntax defines an additional SKYLINE clause in SQL that specifies how the skyline operation should be performed. %In the SKYLINE clause, relevant attributes can be listed. For each attribute, the syntax defines three attribute specifiers, MIN, MAX, and DIFF, that tell the operator how each attribute is to be handled. MIN and MAX denote that the values of the attribute should be minimized or maximized; DIFF denotes that the values should be different.

\subsection{Wireless Broadcast Index}\label{sec:wireless_bcast_index}

%Many excellent studies have been done on improving the efficiency of wireless broadcast systems using indexing techniques. This section considers several popular index allocation techniques and discusses their benefits and drawbacks.

The intuitive technique of no index and one-time index at the beginning of a broadcast cycle has been considered in~\cite{journals/tkde/ImielinskiVB97}. With no index, the length of a cycle is minimized, but the tuning time is the entire cycle since the program is not self-descriptive. With one-time index, the clients are able to filter unwanted data and reduce tuning time, but if a client misses the one-time index, then it will have to wait until the next cycle even if there is useful data in current cycle.

$(1, m)$ indexing, proposed by Imielinski, \emph{et al.} in~\cite{journals/tkde/ImielinskiVB97}, is a mitigation to the problem of one-time index by replicating the entire index every $1/m$ length of the broadcast cycle. The benefit of this index technique is that when a client misses an index segment, it can wait for the next index in the same cycle~\cite{journals/tmc/KuZW08}. The drawback of this technique is the space consumption of replicating the full index several times in the cycle.

Distributed index was also proposed in~\cite{journals/tkde/ImielinskiVB97}. This index also replicates the index in the broadcast cycle, but only a part of the index is replicated. Advantages of this method are (1) an index can be obtained throughout a cycle, (2) reduced bandwidth consumption compared with $(1, m)$, and (3) the method is not limited to any particular index structure.

%\cite{data_on_air} by Imielinski, et al. is an
%early and influential work of data
%indexing for broadcast on air. The chapter defined two
%characteristics of wireless broadcast, \em{tuning time} and \em{access latency}. Tuning time
%defines how long the client has to actively listen on the channel to get all the desired data.
%Access latency define the time when the client issues the query to the time all the desired
%data is received. Tuning time is proportional to power usage of mobile clients. The goal is to
%reduce both tuning time and access latency.

%In addition, the chapter also proposed a few air index techniques to reduce client tuning time.
%The first index proposed is $(1, m)$ index, in which the full index is repeated every $1/m$ of
%the entire broadcast cycle. The index is repeat so that clients that tune into the channel in
%the middle of the broadcast does not have to wait until the next cycle to get the index and
%process request. The second index proposed is the distributed index, in which only a part of
%the index is repeated. Distributed index reduces the overhead of embedding index and improves
%access latency.

%Data filtering based on data signatures was proposed in~\cite{journals/winet/LeeL99}. During data retrieval, the signature of the desired data is compared with the signature of a data segment prepared by the broadcast server. If the signatures match, then the data is downloaded; otherwise it is ignored.
Distributed index for spatial data in error-prone air broadcast was introduced in~\cite{journals/vldb/ZhengLLLS09}. Instead of replication, this paper proposes a distributed index in the broadcast cycle with no duplicate indexes. The proposed solution indexes broadcast spatial data using Hilbert values. Each data record contains an index table that includes the data segments that will be pushed onto the channel in the near future. %A drawback of this approach is the loss of spatial precision due to the use of the space filling curve as the index.


%\section{System Model}

%In our theoretical model, the system consists of the broadcast server and multiple clients. As
%discussed in section 2.1, the server periodically broadcast data in a specific channel. Any client
%that is interested in any of the data serviced by the server tunes into the channel to get the
%desire data. The server contains a set of records as illustrated in Figure 4.

%In this model, we assume a pure broadcast model in which all data are transmitted through downlink
%bandwidth and that there is not uplink bandwidth. Clients must tune into the channel as long as it
%takes to obtain all desire data. Apparently, in order to support efficient query, the server must
%provide data index so that the client can tune in only when the relevant data is to be broadcasted
%In addition, this model contains only one transmission channel. Index must be transmitted on the
%same channel as the data.
